% Custom LaTeX Commands for FDM 3D Printing Paper
% Last Updated: 2026-02-02
%
% Usage: Include in preamble with % Custom LaTeX Commands for FDM 3D Printing Paper
% Last Updated: 2026-02-02
%
% Usage: Include in preamble with % Custom LaTeX Commands for FDM 3D Printing Paper
% Last Updated: 2026-02-02
%
% Usage: Include in preamble with % Custom LaTeX Commands for FDM 3D Printing Paper
% Last Updated: 2026-02-02
%
% Usage: Include in preamble with \input{commands.tex}

%==============================================================================
% Physics Symbols
%==============================================================================

% Position, velocity, acceleration
\newcommand{\vx}{\mathbf{x}}
\newcommand{\vv}{\mathbf{v}}
\newcommand{\va}{\mathbf{a}}
\newcommand{\vj}{\mathbf{j}}
\newcommand{\vr}{\mathbf{r}}

% Vectors
\newcommand{\vF}{\mathbf{F}}
\newcommand{\ve}{\mathbf{e}}
\newcommand{\vu}{\mathbf{u}}
\newcommand{\vz}{\mathbf{z}}

% Time derivatives
\newcommand{\ddot}[1]{\ensuremath{\frac{d^2 #1}{dt^2}}}
\newcommand{\dot}[1]{\ensuremath{\frac{d #1}{dt}}}

% Partial derivatives
\newcommand{\pdd}[2]{\ensuremath{\frac{\partial^2 #1}{\partial #2^2}}}
\newcommand{\pd}[2]{\ensuremath{\frac{\partial #1}{\partial #2}}}

%==============================================================================
% Thermal Symbols
%==============================================================================

\newcommand{\Tnozzle}{T_{\text{nozzle}}}
\newcommand{\Tinterface}{T_{\text{interface}}}
\newcommand{\Tambient}{T_{\text{amb}}}
\newcommand{\Tsurf}{T_{\text{surface}}}

% Critical temperatures
\newcommand{\Tg}{T_g}           % Glass transition
\newcommand{\Tm}{T_m}           % Melting point

% Time constants
\newcommand{\tauheat}{\tau_{\text{heating}}}
\newcommand{\taucool}{\tau_{\text{cooling}}}

%==============================================================================
% Material Properties
%==============================================================================

\newcommand{\Ea}{E_a}          % Activation energy
\newcommand{\Rgas}{R}          % Gas constant
\newcommand{\sigmabulk}{\sigma_{\text{bulk}}}
\newcommand{\sigmaadh}{\sigma_{\text{adh}}}

% PLA properties
\newcommand{\rhoPLA}{1240}     % kg/m^3
\newcommand{\cpPLA}{1200}      % J/(kg·K)
\newcommand{\kPLA}{0.13}       % W/(m·K)
\newcommand{\alphaPLA}{8.7 \times 10^{-8}}  % m^2/s

%==============================================================================
% Dynamic Symbols
%==============================================================================

\newcommand{\omegan}{\omega_n}         % Natural frequency
\newcommand{\omegad}{\omega_d}          % Damped frequency
\newcommand{\zetaan}{\zeta}             % Damping ratio

% Error metrics
\newcommand{\MAE}{\text{MAE}}
\newcommand{\RMSE}{\text{RMSE}}
\newcommand{\Rsq}{R^2}

%==============================================================================
% Math Operators
%==============================================================================

% Common functions
\DeclareMathOperator{\sinc}{sinc}
\DeclareMathOperator{\rect}{rect}
\DeclareMathOperator{\sgn}{sgn}

% Expected value
\newcommand{\E}{\mathbb{E}}

% Norm
\newcommand{\norm}[1]{\left\lVert#1\right\rVert}

% Absolute value
\newcommand{\abs}[1]{\left\lvert#1\right\rvert}

% Floor and ceiling
\newcommand{\floor}[1]{\left\lfloor#1\right\rfloor}
\newcommand{\ceil}[1]{\left\lceil#1\right\rceil}

%==============================================================================
% Equation Environments
%==============================================================================

% Aligned equations with numbers
\newenvironment{alignedeq}
{\begin{aligned}}
{\end{aligned}}

% Cases environment shortcut
\newcommand{\case}[2]{#1 & #2 \\}

%==============================================================================
% Units and Quantities
%==============================================================================

% Units (use siunitx package for proper formatting)
\usepackage{siunitx}

% Common unit commands
\newcommand{\mm}{\si{\milli\meter}}
\newcommand{\cm}{\si{\centi\meter}}
\newcommand{\metersq}{\si{\meter\cubed}}
\newcommand{\mmcubed}{\si{\milli\meter\cubed}}
\newcommand{\degC}{\si{\degreeCelsius}}
\newcommand{\kelvin}{\si{\kelvin}}
\newcommand{\kg}{\si{\kilo\gram}}
\newcommand{\gram}{\si{\gram}}
\newcommand{\second}{\si{\second}}
\newcommand{\minute}{\si{\minute}}
\newcommand{\hour}{\si{\hour}}
\newcommand{\hertz}{\si{\hertz}}
\newcommand{\newton}{\si{\newton}}
\newcommand{\pascal}{\si{\pascal}}
\newcommand{\joule}{\si{\joule}}
\newcommand{\watt}{\si{\watt}}

% Compound units
\newcommand{\nperm}{\si{\newton\per\meter}}
\newcommand{\nspersm}{\si{\newton\second\per\meter}}
\newcommand{\wattpermk}{\si{\watt\per\meter\kelvin}}
\newcommand{\jkpmk}{\si{\joule\per\kilo\gram\kelvin}}

%==============================================================================
% Text Formatting
%==============================================================================

% Emphasis
\newcommand{\keyword}[1]{\textbf{\textit{#1}}}

% Notes
\newcommand{\note}[1]{\textcolor{gray}{\small \textit{#1}}}

% Common abbreviations
\newcommand{\ie}{\textit{i.e.}}
\newcommand{\eg}{\textit{e.g.}}
\newcommand{\etc}{\textit{etc.}}
\newcommand{\vs}{\textit{vs.}}
\newcommand{\etal}{\textit{et al.}}

% Math text
\newcommand{\tr}{^{\mathsf{T}}}     % Transpose
\newcommand{\inv}{^{-1}}             % Inverse

%==============================================================================
% Figure and Table References
%==============================================================================

% Simplified references
\newcommand{\figref}[1]{Figure~\ref{#1}}
\newcommand{\tabref}[1]{Table~\ref{#1}}
\newcommand{\eqref}[1]{Equation~(\ref{#1})}
\newcommand{\secref}[1]{Section~\ref{#1}}

% Multiple references
\newcommand{\Figs}[2]{Figs.~\ref{#1}--\ref{#2}}
\newcommand{\FigsThree}[3]{Figs.~\ref{#1}, \ref{#2}, \ref{#3}}
\newcommand{\Eqs}[2]{Eqs.~(\ref{#1})--(\ref{#2})}

%==============================================================================
% Algorithm Environments
%==============================================================================

\usepackage{algorithm}
\usepackage{algpseudocode}

% Custom algorithm commands
\renewcommand{\algorithmicrequire}{\textbf{Input:}}
\renewcommand{\algorithmicensure}{\textbf{Output:}}
\newcommand{\algcost}[1]{\texttt{#1}}

%==============================================================================
% Code Listings
%==============================================================================

\usepackage{listings}
\usepackage{xcolor}

\definecolor{codegreen}{rgb}{0,0.6,0}
\definecolor{codegray}{rgb}{0.5,0.5,0.5}
\definecolor{codepurple}{rgb}{0.58,0,0.82}
\definecolor{backcolour}{rgb}{0.95,0.95,0.92}

\lstdefinestyle{mystyle}{
    backgroundcolor=\color{backcolour},
    commentstyle=\color{codegreen},
    keywordstyle=\color{magenta},
    numberstyle=\tiny\color{codegray},
    stringstyle=\color{codepurple},
    basicstyle=\ttfamily\footnotesize,
    breakatwhitespace=false,
    breaklines=true,
    captionpos=b,
    keepspaces=true,
    numbers=left,
    numbersep=5pt,
    showspaces=false,
    showstringspaces=false,
    showtabs=false,
    tabsize=2
}

\lstset{style=mystyle}

%==============================================================================
% Theorems and Proofs
%==============================================================================

\usepackage{amsthm}

\newtheorem{theorem}{Theorem}
\newtheorem{lemma}[theorem]{Lemma}
\newtheorem{proposition}[theorem]{Proposition}
\newtheorem{corollary}[theorem]{Corollary}

\theoremstyle{definition}
\newtheorem{definition}{Definition}
\newtheorem{example}{Example}

\theoremstyle{remark}
\newtheorem{remark}{Remark}

%==============================================================================
% TikZ Settings (for diagrams)
%==============================================================================

\usepackage{tikz}
\usetikzlibrary{arrows, positioning, shapes, calc}

% Common TikZ styles
\tikzset{
    block/.style = {
        draw,
        rectangle,
        minimum height=3em,
        minimum width=4em,
        align=center,
        fill=blue!10
    },
    sum/.style = {
        draw,
        circle,
        minimum size=6mm,
        fill=yellow!10
    },
    input/.style = {
        coordinate
    },
    output/.style = {
        coordinate
    },
    pinstyle/.style = {
        pin edge={to-,thin,black}
    }
}

%==============================================================================
% Tables
%==============================================================================

\usepackage{booktabs}  % For professional tables

% Custom table commands
\newcommand{\topline}{\hline\hline}
\newcommand{\midline}{\hline}
\newcommand{\bottomline}{\hline\hline}

%==============================================================================
% Bibliography
%==============================================================================

\newcommand{\citep}[1]{(\cite{#1})}
\newcommand{\citet}[1]{\citeauthor{#1} (\citeyear{#1})}

%==============================================================================
% Abbreviations for This Paper
%==============================================================================

% Model names
\newcommand{\ModelName}{TrajectoryErrorTransformer}
\newcommand{\PINN}{Physics-Informed Neural Network}
\newcommand{\FDM}{Fused Deposition Modeling}

% Dataset names
\newcommand{\DatasetTrain}{\texttt{train}}
\newcommand{\DatasetVal}{\texttt{val}}
\newcommand{\DatasetTest}{\texttt{test}}

%==============================================================================
% Page Layout (for IEEE format)
%==============================================================================

% Column width for IEEE format
\newcommand{\columnwidth}{3.5in}  % For single column figures
\newcommand{\doublecolumnwidth}{7in}  % For double column figures

% Figure width helper
\newcommand{\onecol}[1]{\begin{column}{\columnwidth}#1\end{column}}
\newcommand{\twocol}[1]{\begin{column}{\doublecolumnwidth}#1\end{column}}

%==============================================================================
% Misc Helpers
%==============================================================================

% Circled numbers
\newcommand{\circled}[1]{\tikz[baseline=(char.base)]{
    \node[shape=circle,draw,inner sep=1pt] (char) {\small #1};}}

% Checkmark
\newcommand{\checkmark}{\tikz\fill[scale=0.4, color=green!60!black](0,.35) -- (.25,0) -- (1,.7) -- (.25,.15) -- cycle;}

% X mark
\newcommand{\xmark}{\tikz\fill[scale=0.4, color=red] (0,0) -- (1,1) -- (0,1) -- (1,0) -- cycle;}

% Status indicators
\newcommand{\status}[2]{%
    \ifstrequal{#1}{pass}{\textcolor{green}{\checkmark}}%
    {\ifstrequal{#1}{fail}{\textcolor{red}{\xmark}}%
    {\ifstrequal{#1}{warn}{\textcolor{orange}{\textbf{!}}}%
    {#1}}}%
}

%==============================================================================
% Draft Watermark
%==============================================================================

\newcommand{\draftwatermark}{
    \begin{tikzpicture}[remember picture, overlay]
        \node[rotate=30, scale=10, text opacity=0.05] at (current page.center) {
            \textbf{DRAFT}
        };
    \end{tikzpicture}
}

% Uncomment to add watermark:
% \draftwatermark

%==============================================================================
% Print/Online Differences
%==============================================================================

% Use this to show different content for print vs online
\newcommand{\onlineonly}[1]{\ifmode{online}{#1}{}}
\newcommand{\printonly}[1]{\ifmode{print}{#1}{}}

%==============================================================================
% Common Math Shortcuts
%==============================================================================

% Derivatives
\newcommand{\ddx}[1]{\frac{d#1}{dx}}
\newcommand{\pdx}[1]{\frac{\partial #1}{\partial x}}
\newcommand{\ddt}[1]{\frac{d#1}{dt}}
\newcommand{\pdt}[1]{\frac{\partial #1}{\partial t}}

% Integrals
\newcommand{\intf}{\int_{-\infty}^{\infty}}
\newcommand{\intzero}{\int_{0}^{\infty}}

% Limits
\newcommand{\limn}{\lim_{n \to \infty}}
\newcommand{\limx}{\lim_{x \to \infty}}
\newcommand{\limzero}{\lim_{x \to 0}}

% Big parentheses
\newcommand{\bigp}[1]{\bigl(#1\bigr)}
\newcommand{\Bigp}[1]{\Bigl(#1\Bigr)}
\newcommand{\biggp}[1]{\biggl(#1\biggr)}
\newcommand{\Biggp}[1]{\Biggl(#1\Biggr)}

% Big brackets
\newcommand{\bigb}[1]{\bigl[#1\bigr]}
\newcommand{\Bigb}[1]{\Bigl[#1\Bigr]}
\newcommand{\biggb}[1]{\biggl[#1\biggr]}
\newcommand{\Biggb}[1]{\Biggl[#1\Biggr]}

%==============================================================================
% End of Custom Commands
%==============================================================================


%==============================================================================
% Physics Symbols
%==============================================================================

% Position, velocity, acceleration
\newcommand{\vx}{\mathbf{x}}
\newcommand{\vv}{\mathbf{v}}
\newcommand{\va}{\mathbf{a}}
\newcommand{\vj}{\mathbf{j}}
\newcommand{\vr}{\mathbf{r}}

% Vectors
\newcommand{\vF}{\mathbf{F}}
\newcommand{\ve}{\mathbf{e}}
\newcommand{\vu}{\mathbf{u}}
\newcommand{\vz}{\mathbf{z}}

% Time derivatives
\newcommand{\ddot}[1]{\ensuremath{\frac{d^2 #1}{dt^2}}}
\newcommand{\dot}[1]{\ensuremath{\frac{d #1}{dt}}}

% Partial derivatives
\newcommand{\pdd}[2]{\ensuremath{\frac{\partial^2 #1}{\partial #2^2}}}
\newcommand{\pd}[2]{\ensuremath{\frac{\partial #1}{\partial #2}}}

%==============================================================================
% Thermal Symbols
%==============================================================================

\newcommand{\Tnozzle}{T_{\text{nozzle}}}
\newcommand{\Tinterface}{T_{\text{interface}}}
\newcommand{\Tambient}{T_{\text{amb}}}
\newcommand{\Tsurf}{T_{\text{surface}}}

% Critical temperatures
\newcommand{\Tg}{T_g}           % Glass transition
\newcommand{\Tm}{T_m}           % Melting point

% Time constants
\newcommand{\tauheat}{\tau_{\text{heating}}}
\newcommand{\taucool}{\tau_{\text{cooling}}}

%==============================================================================
% Material Properties
%==============================================================================

\newcommand{\Ea}{E_a}          % Activation energy
\newcommand{\Rgas}{R}          % Gas constant
\newcommand{\sigmabulk}{\sigma_{\text{bulk}}}
\newcommand{\sigmaadh}{\sigma_{\text{adh}}}

% PLA properties
\newcommand{\rhoPLA}{1240}     % kg/m^3
\newcommand{\cpPLA}{1200}      % J/(kg·K)
\newcommand{\kPLA}{0.13}       % W/(m·K)
\newcommand{\alphaPLA}{8.7 \times 10^{-8}}  % m^2/s

%==============================================================================
% Dynamic Symbols
%==============================================================================

\newcommand{\omegan}{\omega_n}         % Natural frequency
\newcommand{\omegad}{\omega_d}          % Damped frequency
\newcommand{\zetaan}{\zeta}             % Damping ratio

% Error metrics
\newcommand{\MAE}{\text{MAE}}
\newcommand{\RMSE}{\text{RMSE}}
\newcommand{\Rsq}{R^2}

%==============================================================================
% Math Operators
%==============================================================================

% Common functions
\DeclareMathOperator{\sinc}{sinc}
\DeclareMathOperator{\rect}{rect}
\DeclareMathOperator{\sgn}{sgn}

% Expected value
\newcommand{\E}{\mathbb{E}}

% Norm
\newcommand{\norm}[1]{\left\lVert#1\right\rVert}

% Absolute value
\newcommand{\abs}[1]{\left\lvert#1\right\rvert}

% Floor and ceiling
\newcommand{\floor}[1]{\left\lfloor#1\right\rfloor}
\newcommand{\ceil}[1]{\left\lceil#1\right\rceil}

%==============================================================================
% Equation Environments
%==============================================================================

% Aligned equations with numbers
\newenvironment{alignedeq}
{\begin{aligned}}
{\end{aligned}}

% Cases environment shortcut
\newcommand{\case}[2]{#1 & #2 \\}

%==============================================================================
% Units and Quantities
%==============================================================================

% Units (use siunitx package for proper formatting)
\usepackage{siunitx}

% Common unit commands
\newcommand{\mm}{\si{\milli\meter}}
\newcommand{\cm}{\si{\centi\meter}}
\newcommand{\metersq}{\si{\meter\cubed}}
\newcommand{\mmcubed}{\si{\milli\meter\cubed}}
\newcommand{\degC}{\si{\degreeCelsius}}
\newcommand{\kelvin}{\si{\kelvin}}
\newcommand{\kg}{\si{\kilo\gram}}
\newcommand{\gram}{\si{\gram}}
\newcommand{\second}{\si{\second}}
\newcommand{\minute}{\si{\minute}}
\newcommand{\hour}{\si{\hour}}
\newcommand{\hertz}{\si{\hertz}}
\newcommand{\newton}{\si{\newton}}
\newcommand{\pascal}{\si{\pascal}}
\newcommand{\joule}{\si{\joule}}
\newcommand{\watt}{\si{\watt}}

% Compound units
\newcommand{\nperm}{\si{\newton\per\meter}}
\newcommand{\nspersm}{\si{\newton\second\per\meter}}
\newcommand{\wattpermk}{\si{\watt\per\meter\kelvin}}
\newcommand{\jkpmk}{\si{\joule\per\kilo\gram\kelvin}}

%==============================================================================
% Text Formatting
%==============================================================================

% Emphasis
\newcommand{\keyword}[1]{\textbf{\textit{#1}}}

% Notes
\newcommand{\note}[1]{\textcolor{gray}{\small \textit{#1}}}

% Common abbreviations
\newcommand{\ie}{\textit{i.e.}}
\newcommand{\eg}{\textit{e.g.}}
\newcommand{\etc}{\textit{etc.}}
\newcommand{\vs}{\textit{vs.}}
\newcommand{\etal}{\textit{et al.}}

% Math text
\newcommand{\tr}{^{\mathsf{T}}}     % Transpose
\newcommand{\inv}{^{-1}}             % Inverse

%==============================================================================
% Figure and Table References
%==============================================================================

% Simplified references
\newcommand{\figref}[1]{Figure~\ref{#1}}
\newcommand{\tabref}[1]{Table~\ref{#1}}
\newcommand{\eqref}[1]{Equation~(\ref{#1})}
\newcommand{\secref}[1]{Section~\ref{#1}}

% Multiple references
\newcommand{\Figs}[2]{Figs.~\ref{#1}--\ref{#2}}
\newcommand{\FigsThree}[3]{Figs.~\ref{#1}, \ref{#2}, \ref{#3}}
\newcommand{\Eqs}[2]{Eqs.~(\ref{#1})--(\ref{#2})}

%==============================================================================
% Algorithm Environments
%==============================================================================

\usepackage{algorithm}
\usepackage{algpseudocode}

% Custom algorithm commands
\renewcommand{\algorithmicrequire}{\textbf{Input:}}
\renewcommand{\algorithmicensure}{\textbf{Output:}}
\newcommand{\algcost}[1]{\texttt{#1}}

%==============================================================================
% Code Listings
%==============================================================================

\usepackage{listings}
\usepackage{xcolor}

\definecolor{codegreen}{rgb}{0,0.6,0}
\definecolor{codegray}{rgb}{0.5,0.5,0.5}
\definecolor{codepurple}{rgb}{0.58,0,0.82}
\definecolor{backcolour}{rgb}{0.95,0.95,0.92}

\lstdefinestyle{mystyle}{
    backgroundcolor=\color{backcolour},
    commentstyle=\color{codegreen},
    keywordstyle=\color{magenta},
    numberstyle=\tiny\color{codegray},
    stringstyle=\color{codepurple},
    basicstyle=\ttfamily\footnotesize,
    breakatwhitespace=false,
    breaklines=true,
    captionpos=b,
    keepspaces=true,
    numbers=left,
    numbersep=5pt,
    showspaces=false,
    showstringspaces=false,
    showtabs=false,
    tabsize=2
}

\lstset{style=mystyle}

%==============================================================================
% Theorems and Proofs
%==============================================================================

\usepackage{amsthm}

\newtheorem{theorem}{Theorem}
\newtheorem{lemma}[theorem]{Lemma}
\newtheorem{proposition}[theorem]{Proposition}
\newtheorem{corollary}[theorem]{Corollary}

\theoremstyle{definition}
\newtheorem{definition}{Definition}
\newtheorem{example}{Example}

\theoremstyle{remark}
\newtheorem{remark}{Remark}

%==============================================================================
% TikZ Settings (for diagrams)
%==============================================================================

\usepackage{tikz}
\usetikzlibrary{arrows, positioning, shapes, calc}

% Common TikZ styles
\tikzset{
    block/.style = {
        draw,
        rectangle,
        minimum height=3em,
        minimum width=4em,
        align=center,
        fill=blue!10
    },
    sum/.style = {
        draw,
        circle,
        minimum size=6mm,
        fill=yellow!10
    },
    input/.style = {
        coordinate
    },
    output/.style = {
        coordinate
    },
    pinstyle/.style = {
        pin edge={to-,thin,black}
    }
}

%==============================================================================
% Tables
%==============================================================================

\usepackage{booktabs}  % For professional tables

% Custom table commands
\newcommand{\topline}{\hline\hline}
\newcommand{\midline}{\hline}
\newcommand{\bottomline}{\hline\hline}

%==============================================================================
% Bibliography
%==============================================================================

\newcommand{\citep}[1]{(\cite{#1})}
\newcommand{\citet}[1]{\citeauthor{#1} (\citeyear{#1})}

%==============================================================================
% Abbreviations for This Paper
%==============================================================================

% Model names
\newcommand{\ModelName}{TrajectoryErrorTransformer}
\newcommand{\PINN}{Physics-Informed Neural Network}
\newcommand{\FDM}{Fused Deposition Modeling}

% Dataset names
\newcommand{\DatasetTrain}{\texttt{train}}
\newcommand{\DatasetVal}{\texttt{val}}
\newcommand{\DatasetTest}{\texttt{test}}

%==============================================================================
% Page Layout (for IEEE format)
%==============================================================================

% Column width for IEEE format
\newcommand{\columnwidth}{3.5in}  % For single column figures
\newcommand{\doublecolumnwidth}{7in}  % For double column figures

% Figure width helper
\newcommand{\onecol}[1]{\begin{column}{\columnwidth}#1\end{column}}
\newcommand{\twocol}[1]{\begin{column}{\doublecolumnwidth}#1\end{column}}

%==============================================================================
% Misc Helpers
%==============================================================================

% Circled numbers
\newcommand{\circled}[1]{\tikz[baseline=(char.base)]{
    \node[shape=circle,draw,inner sep=1pt] (char) {\small #1};}}

% Checkmark
\newcommand{\checkmark}{\tikz\fill[scale=0.4, color=green!60!black](0,.35) -- (.25,0) -- (1,.7) -- (.25,.15) -- cycle;}

% X mark
\newcommand{\xmark}{\tikz\fill[scale=0.4, color=red] (0,0) -- (1,1) -- (0,1) -- (1,0) -- cycle;}

% Status indicators
\newcommand{\status}[2]{%
    \ifstrequal{#1}{pass}{\textcolor{green}{\checkmark}}%
    {\ifstrequal{#1}{fail}{\textcolor{red}{\xmark}}%
    {\ifstrequal{#1}{warn}{\textcolor{orange}{\textbf{!}}}%
    {#1}}}%
}

%==============================================================================
% Draft Watermark
%==============================================================================

\newcommand{\draftwatermark}{
    \begin{tikzpicture}[remember picture, overlay]
        \node[rotate=30, scale=10, text opacity=0.05] at (current page.center) {
            \textbf{DRAFT}
        };
    \end{tikzpicture}
}

% Uncomment to add watermark:
% \draftwatermark

%==============================================================================
% Print/Online Differences
%==============================================================================

% Use this to show different content for print vs online
\newcommand{\onlineonly}[1]{\ifmode{online}{#1}{}}
\newcommand{\printonly}[1]{\ifmode{print}{#1}{}}

%==============================================================================
% Common Math Shortcuts
%==============================================================================

% Derivatives
\newcommand{\ddx}[1]{\frac{d#1}{dx}}
\newcommand{\pdx}[1]{\frac{\partial #1}{\partial x}}
\newcommand{\ddt}[1]{\frac{d#1}{dt}}
\newcommand{\pdt}[1]{\frac{\partial #1}{\partial t}}

% Integrals
\newcommand{\intf}{\int_{-\infty}^{\infty}}
\newcommand{\intzero}{\int_{0}^{\infty}}

% Limits
\newcommand{\limn}{\lim_{n \to \infty}}
\newcommand{\limx}{\lim_{x \to \infty}}
\newcommand{\limzero}{\lim_{x \to 0}}

% Big parentheses
\newcommand{\bigp}[1]{\bigl(#1\bigr)}
\newcommand{\Bigp}[1]{\Bigl(#1\Bigr)}
\newcommand{\biggp}[1]{\biggl(#1\biggr)}
\newcommand{\Biggp}[1]{\Biggl(#1\Biggr)}

% Big brackets
\newcommand{\bigb}[1]{\bigl[#1\bigr]}
\newcommand{\Bigb}[1]{\Bigl[#1\Bigr]}
\newcommand{\biggb}[1]{\biggl[#1\biggr]}
\newcommand{\Biggb}[1]{\Biggl[#1\Biggr]}

%==============================================================================
% End of Custom Commands
%==============================================================================


%==============================================================================
% Physics Symbols
%==============================================================================

% Position, velocity, acceleration
\newcommand{\vx}{\mathbf{x}}
\newcommand{\vv}{\mathbf{v}}
\newcommand{\va}{\mathbf{a}}
\newcommand{\vj}{\mathbf{j}}
\newcommand{\vr}{\mathbf{r}}

% Vectors
\newcommand{\vF}{\mathbf{F}}
\newcommand{\ve}{\mathbf{e}}
\newcommand{\vu}{\mathbf{u}}
\newcommand{\vz}{\mathbf{z}}

% Time derivatives
\newcommand{\ddot}[1]{\ensuremath{\frac{d^2 #1}{dt^2}}}
\newcommand{\dot}[1]{\ensuremath{\frac{d #1}{dt}}}

% Partial derivatives
\newcommand{\pdd}[2]{\ensuremath{\frac{\partial^2 #1}{\partial #2^2}}}
\newcommand{\pd}[2]{\ensuremath{\frac{\partial #1}{\partial #2}}}

%==============================================================================
% Thermal Symbols
%==============================================================================

\newcommand{\Tnozzle}{T_{\text{nozzle}}}
\newcommand{\Tinterface}{T_{\text{interface}}}
\newcommand{\Tambient}{T_{\text{amb}}}
\newcommand{\Tsurf}{T_{\text{surface}}}

% Critical temperatures
\newcommand{\Tg}{T_g}           % Glass transition
\newcommand{\Tm}{T_m}           % Melting point

% Time constants
\newcommand{\tauheat}{\tau_{\text{heating}}}
\newcommand{\taucool}{\tau_{\text{cooling}}}

%==============================================================================
% Material Properties
%==============================================================================

\newcommand{\Ea}{E_a}          % Activation energy
\newcommand{\Rgas}{R}          % Gas constant
\newcommand{\sigmabulk}{\sigma_{\text{bulk}}}
\newcommand{\sigmaadh}{\sigma_{\text{adh}}}

% PLA properties
\newcommand{\rhoPLA}{1240}     % kg/m^3
\newcommand{\cpPLA}{1200}      % J/(kg·K)
\newcommand{\kPLA}{0.13}       % W/(m·K)
\newcommand{\alphaPLA}{8.7 \times 10^{-8}}  % m^2/s

%==============================================================================
% Dynamic Symbols
%==============================================================================

\newcommand{\omegan}{\omega_n}         % Natural frequency
\newcommand{\omegad}{\omega_d}          % Damped frequency
\newcommand{\zetaan}{\zeta}             % Damping ratio

% Error metrics
\newcommand{\MAE}{\text{MAE}}
\newcommand{\RMSE}{\text{RMSE}}
\newcommand{\Rsq}{R^2}

%==============================================================================
% Math Operators
%==============================================================================

% Common functions
\DeclareMathOperator{\sinc}{sinc}
\DeclareMathOperator{\rect}{rect}
\DeclareMathOperator{\sgn}{sgn}

% Expected value
\newcommand{\E}{\mathbb{E}}

% Norm
\newcommand{\norm}[1]{\left\lVert#1\right\rVert}

% Absolute value
\newcommand{\abs}[1]{\left\lvert#1\right\rvert}

% Floor and ceiling
\newcommand{\floor}[1]{\left\lfloor#1\right\rfloor}
\newcommand{\ceil}[1]{\left\lceil#1\right\rceil}

%==============================================================================
% Equation Environments
%==============================================================================

% Aligned equations with numbers
\newenvironment{alignedeq}
{\begin{aligned}}
{\end{aligned}}

% Cases environment shortcut
\newcommand{\case}[2]{#1 & #2 \\}

%==============================================================================
% Units and Quantities
%==============================================================================

% Units (use siunitx package for proper formatting)
\usepackage{siunitx}

% Common unit commands
\newcommand{\mm}{\si{\milli\meter}}
\newcommand{\cm}{\si{\centi\meter}}
\newcommand{\metersq}{\si{\meter\cubed}}
\newcommand{\mmcubed}{\si{\milli\meter\cubed}}
\newcommand{\degC}{\si{\degreeCelsius}}
\newcommand{\kelvin}{\si{\kelvin}}
\newcommand{\kg}{\si{\kilo\gram}}
\newcommand{\gram}{\si{\gram}}
\newcommand{\second}{\si{\second}}
\newcommand{\minute}{\si{\minute}}
\newcommand{\hour}{\si{\hour}}
\newcommand{\hertz}{\si{\hertz}}
\newcommand{\newton}{\si{\newton}}
\newcommand{\pascal}{\si{\pascal}}
\newcommand{\joule}{\si{\joule}}
\newcommand{\watt}{\si{\watt}}

% Compound units
\newcommand{\nperm}{\si{\newton\per\meter}}
\newcommand{\nspersm}{\si{\newton\second\per\meter}}
\newcommand{\wattpermk}{\si{\watt\per\meter\kelvin}}
\newcommand{\jkpmk}{\si{\joule\per\kilo\gram\kelvin}}

%==============================================================================
% Text Formatting
%==============================================================================

% Emphasis
\newcommand{\keyword}[1]{\textbf{\textit{#1}}}

% Notes
\newcommand{\note}[1]{\textcolor{gray}{\small \textit{#1}}}

% Common abbreviations
\newcommand{\ie}{\textit{i.e.}}
\newcommand{\eg}{\textit{e.g.}}
\newcommand{\etc}{\textit{etc.}}
\newcommand{\vs}{\textit{vs.}}
\newcommand{\etal}{\textit{et al.}}

% Math text
\newcommand{\tr}{^{\mathsf{T}}}     % Transpose
\newcommand{\inv}{^{-1}}             % Inverse

%==============================================================================
% Figure and Table References
%==============================================================================

% Simplified references
\newcommand{\figref}[1]{Figure~\ref{#1}}
\newcommand{\tabref}[1]{Table~\ref{#1}}
\newcommand{\eqref}[1]{Equation~(\ref{#1})}
\newcommand{\secref}[1]{Section~\ref{#1}}

% Multiple references
\newcommand{\Figs}[2]{Figs.~\ref{#1}--\ref{#2}}
\newcommand{\FigsThree}[3]{Figs.~\ref{#1}, \ref{#2}, \ref{#3}}
\newcommand{\Eqs}[2]{Eqs.~(\ref{#1})--(\ref{#2})}

%==============================================================================
% Algorithm Environments
%==============================================================================

\usepackage{algorithm}
\usepackage{algpseudocode}

% Custom algorithm commands
\renewcommand{\algorithmicrequire}{\textbf{Input:}}
\renewcommand{\algorithmicensure}{\textbf{Output:}}
\newcommand{\algcost}[1]{\texttt{#1}}

%==============================================================================
% Code Listings
%==============================================================================

\usepackage{listings}
\usepackage{xcolor}

\definecolor{codegreen}{rgb}{0,0.6,0}
\definecolor{codegray}{rgb}{0.5,0.5,0.5}
\definecolor{codepurple}{rgb}{0.58,0,0.82}
\definecolor{backcolour}{rgb}{0.95,0.95,0.92}

\lstdefinestyle{mystyle}{
    backgroundcolor=\color{backcolour},
    commentstyle=\color{codegreen},
    keywordstyle=\color{magenta},
    numberstyle=\tiny\color{codegray},
    stringstyle=\color{codepurple},
    basicstyle=\ttfamily\footnotesize,
    breakatwhitespace=false,
    breaklines=true,
    captionpos=b,
    keepspaces=true,
    numbers=left,
    numbersep=5pt,
    showspaces=false,
    showstringspaces=false,
    showtabs=false,
    tabsize=2
}

\lstset{style=mystyle}

%==============================================================================
% Theorems and Proofs
%==============================================================================

\usepackage{amsthm}

\newtheorem{theorem}{Theorem}
\newtheorem{lemma}[theorem]{Lemma}
\newtheorem{proposition}[theorem]{Proposition}
\newtheorem{corollary}[theorem]{Corollary}

\theoremstyle{definition}
\newtheorem{definition}{Definition}
\newtheorem{example}{Example}

\theoremstyle{remark}
\newtheorem{remark}{Remark}

%==============================================================================
% TikZ Settings (for diagrams)
%==============================================================================

\usepackage{tikz}
\usetikzlibrary{arrows, positioning, shapes, calc}

% Common TikZ styles
\tikzset{
    block/.style = {
        draw,
        rectangle,
        minimum height=3em,
        minimum width=4em,
        align=center,
        fill=blue!10
    },
    sum/.style = {
        draw,
        circle,
        minimum size=6mm,
        fill=yellow!10
    },
    input/.style = {
        coordinate
    },
    output/.style = {
        coordinate
    },
    pinstyle/.style = {
        pin edge={to-,thin,black}
    }
}

%==============================================================================
% Tables
%==============================================================================

\usepackage{booktabs}  % For professional tables

% Custom table commands
\newcommand{\topline}{\hline\hline}
\newcommand{\midline}{\hline}
\newcommand{\bottomline}{\hline\hline}

%==============================================================================
% Bibliography
%==============================================================================

\newcommand{\citep}[1]{(\cite{#1})}
\newcommand{\citet}[1]{\citeauthor{#1} (\citeyear{#1})}

%==============================================================================
% Abbreviations for This Paper
%==============================================================================

% Model names
\newcommand{\ModelName}{TrajectoryErrorTransformer}
\newcommand{\PINN}{Physics-Informed Neural Network}
\newcommand{\FDM}{Fused Deposition Modeling}

% Dataset names
\newcommand{\DatasetTrain}{\texttt{train}}
\newcommand{\DatasetVal}{\texttt{val}}
\newcommand{\DatasetTest}{\texttt{test}}

%==============================================================================
% Page Layout (for IEEE format)
%==============================================================================

% Column width for IEEE format
\newcommand{\columnwidth}{3.5in}  % For single column figures
\newcommand{\doublecolumnwidth}{7in}  % For double column figures

% Figure width helper
\newcommand{\onecol}[1]{\begin{column}{\columnwidth}#1\end{column}}
\newcommand{\twocol}[1]{\begin{column}{\doublecolumnwidth}#1\end{column}}

%==============================================================================
% Misc Helpers
%==============================================================================

% Circled numbers
\newcommand{\circled}[1]{\tikz[baseline=(char.base)]{
    \node[shape=circle,draw,inner sep=1pt] (char) {\small #1};}}

% Checkmark
\newcommand{\checkmark}{\tikz\fill[scale=0.4, color=green!60!black](0,.35) -- (.25,0) -- (1,.7) -- (.25,.15) -- cycle;}

% X mark
\newcommand{\xmark}{\tikz\fill[scale=0.4, color=red] (0,0) -- (1,1) -- (0,1) -- (1,0) -- cycle;}

% Status indicators
\newcommand{\status}[2]{%
    \ifstrequal{#1}{pass}{\textcolor{green}{\checkmark}}%
    {\ifstrequal{#1}{fail}{\textcolor{red}{\xmark}}%
    {\ifstrequal{#1}{warn}{\textcolor{orange}{\textbf{!}}}%
    {#1}}}%
}

%==============================================================================
% Draft Watermark
%==============================================================================

\newcommand{\draftwatermark}{
    \begin{tikzpicture}[remember picture, overlay]
        \node[rotate=30, scale=10, text opacity=0.05] at (current page.center) {
            \textbf{DRAFT}
        };
    \end{tikzpicture}
}

% Uncomment to add watermark:
% \draftwatermark

%==============================================================================
% Print/Online Differences
%==============================================================================

% Use this to show different content for print vs online
\newcommand{\onlineonly}[1]{\ifmode{online}{#1}{}}
\newcommand{\printonly}[1]{\ifmode{print}{#1}{}}

%==============================================================================
% Common Math Shortcuts
%==============================================================================

% Derivatives
\newcommand{\ddx}[1]{\frac{d#1}{dx}}
\newcommand{\pdx}[1]{\frac{\partial #1}{\partial x}}
\newcommand{\ddt}[1]{\frac{d#1}{dt}}
\newcommand{\pdt}[1]{\frac{\partial #1}{\partial t}}

% Integrals
\newcommand{\intf}{\int_{-\infty}^{\infty}}
\newcommand{\intzero}{\int_{0}^{\infty}}

% Limits
\newcommand{\limn}{\lim_{n \to \infty}}
\newcommand{\limx}{\lim_{x \to \infty}}
\newcommand{\limzero}{\lim_{x \to 0}}

% Big parentheses
\newcommand{\bigp}[1]{\bigl(#1\bigr)}
\newcommand{\Bigp}[1]{\Bigl(#1\Bigr)}
\newcommand{\biggp}[1]{\biggl(#1\biggr)}
\newcommand{\Biggp}[1]{\Biggl(#1\Biggr)}

% Big brackets
\newcommand{\bigb}[1]{\bigl[#1\bigr]}
\newcommand{\Bigb}[1]{\Bigl[#1\Bigr]}
\newcommand{\biggb}[1]{\biggl[#1\biggr]}
\newcommand{\Biggb}[1]{\Biggl[#1\Biggr]}

%==============================================================================
% End of Custom Commands
%==============================================================================


%==============================================================================
% Physics Symbols
%==============================================================================

% Position, velocity, acceleration
\newcommand{\vx}{\mathbf{x}}
\newcommand{\vv}{\mathbf{v}}
\newcommand{\va}{\mathbf{a}}
\newcommand{\vj}{\mathbf{j}}
\newcommand{\vr}{\mathbf{r}}

% Vectors
\newcommand{\vF}{\mathbf{F}}
\newcommand{\ve}{\mathbf{e}}
\newcommand{\vu}{\mathbf{u}}
\newcommand{\vz}{\mathbf{z}}

% Time derivatives
\newcommand{\ddot}[1]{\ensuremath{\frac{d^2 #1}{dt^2}}}
\newcommand{\dot}[1]{\ensuremath{\frac{d #1}{dt}}}

% Partial derivatives
\newcommand{\pdd}[2]{\ensuremath{\frac{\partial^2 #1}{\partial #2^2}}}
\newcommand{\pd}[2]{\ensuremath{\frac{\partial #1}{\partial #2}}}

%==============================================================================
% Thermal Symbols
%==============================================================================

\newcommand{\Tnozzle}{T_{\text{nozzle}}}
\newcommand{\Tinterface}{T_{\text{interface}}}
\newcommand{\Tambient}{T_{\text{amb}}}
\newcommand{\Tsurf}{T_{\text{surface}}}

% Critical temperatures
\newcommand{\Tg}{T_g}           % Glass transition
\newcommand{\Tm}{T_m}           % Melting point

% Time constants
\newcommand{\tauheat}{\tau_{\text{heating}}}
\newcommand{\taucool}{\tau_{\text{cooling}}}

%==============================================================================
% Material Properties
%==============================================================================

\newcommand{\Ea}{E_a}          % Activation energy
\newcommand{\Rgas}{R}          % Gas constant
\newcommand{\sigmabulk}{\sigma_{\text{bulk}}}
\newcommand{\sigmaadh}{\sigma_{\text{adh}}}

% PLA properties
\newcommand{\rhoPLA}{1240}     % kg/m^3
\newcommand{\cpPLA}{1200}      % J/(kg·K)
\newcommand{\kPLA}{0.13}       % W/(m·K)
\newcommand{\alphaPLA}{8.7 \times 10^{-8}}  % m^2/s

%==============================================================================
% Dynamic Symbols
%==============================================================================

\newcommand{\omegan}{\omega_n}         % Natural frequency
\newcommand{\omegad}{\omega_d}          % Damped frequency
\newcommand{\zetaan}{\zeta}             % Damping ratio

% Error metrics
\newcommand{\MAE}{\text{MAE}}
\newcommand{\RMSE}{\text{RMSE}}
\newcommand{\Rsq}{R^2}

%==============================================================================
% Math Operators
%==============================================================================

% Common functions
\DeclareMathOperator{\sinc}{sinc}
\DeclareMathOperator{\rect}{rect}
\DeclareMathOperator{\sgn}{sgn}

% Expected value
\newcommand{\E}{\mathbb{E}}

% Norm
\newcommand{\norm}[1]{\left\lVert#1\right\rVert}

% Absolute value
\newcommand{\abs}[1]{\left\lvert#1\right\rvert}

% Floor and ceiling
\newcommand{\floor}[1]{\left\lfloor#1\right\rfloor}
\newcommand{\ceil}[1]{\left\lceil#1\right\rceil}

%==============================================================================
% Equation Environments
%==============================================================================

% Aligned equations with numbers
\newenvironment{alignedeq}
{\begin{aligned}}
{\end{aligned}}

% Cases environment shortcut
\newcommand{\case}[2]{#1 & #2 \\}

%==============================================================================
% Units and Quantities
%==============================================================================

% Units (use siunitx package for proper formatting)
\usepackage{siunitx}

% Common unit commands
\newcommand{\mm}{\si{\milli\meter}}
\newcommand{\cm}{\si{\centi\meter}}
\newcommand{\metersq}{\si{\meter\cubed}}
\newcommand{\mmcubed}{\si{\milli\meter\cubed}}
\newcommand{\degC}{\si{\degreeCelsius}}
\newcommand{\kelvin}{\si{\kelvin}}
\newcommand{\kg}{\si{\kilo\gram}}
\newcommand{\gram}{\si{\gram}}
\newcommand{\second}{\si{\second}}
\newcommand{\minute}{\si{\minute}}
\newcommand{\hour}{\si{\hour}}
\newcommand{\hertz}{\si{\hertz}}
\newcommand{\newton}{\si{\newton}}
\newcommand{\pascal}{\si{\pascal}}
\newcommand{\joule}{\si{\joule}}
\newcommand{\watt}{\si{\watt}}

% Compound units
\newcommand{\nperm}{\si{\newton\per\meter}}
\newcommand{\nspersm}{\si{\newton\second\per\meter}}
\newcommand{\wattpermk}{\si{\watt\per\meter\kelvin}}
\newcommand{\jkpmk}{\si{\joule\per\kilo\gram\kelvin}}

%==============================================================================
% Text Formatting
%==============================================================================

% Emphasis
\newcommand{\keyword}[1]{\textbf{\textit{#1}}}

% Notes
\newcommand{\note}[1]{\textcolor{gray}{\small \textit{#1}}}

% Common abbreviations
\newcommand{\ie}{\textit{i.e.}}
\newcommand{\eg}{\textit{e.g.}}
\newcommand{\etc}{\textit{etc.}}
\newcommand{\vs}{\textit{vs.}}
\newcommand{\etal}{\textit{et al.}}

% Math text
\newcommand{\tr}{^{\mathsf{T}}}     % Transpose
\newcommand{\inv}{^{-1}}             % Inverse

%==============================================================================
% Figure and Table References
%==============================================================================

% Simplified references
\newcommand{\figref}[1]{Figure~\ref{#1}}
\newcommand{\tabref}[1]{Table~\ref{#1}}
\newcommand{\eqref}[1]{Equation~(\ref{#1})}
\newcommand{\secref}[1]{Section~\ref{#1}}

% Multiple references
\newcommand{\Figs}[2]{Figs.~\ref{#1}--\ref{#2}}
\newcommand{\FigsThree}[3]{Figs.~\ref{#1}, \ref{#2}, \ref{#3}}
\newcommand{\Eqs}[2]{Eqs.~(\ref{#1})--(\ref{#2})}

%==============================================================================
% Algorithm Environments
%==============================================================================

\usepackage{algorithm}
\usepackage{algpseudocode}

% Custom algorithm commands
\renewcommand{\algorithmicrequire}{\textbf{Input:}}
\renewcommand{\algorithmicensure}{\textbf{Output:}}
\newcommand{\algcost}[1]{\texttt{#1}}

%==============================================================================
% Code Listings
%==============================================================================

\usepackage{listings}
\usepackage{xcolor}

\definecolor{codegreen}{rgb}{0,0.6,0}
\definecolor{codegray}{rgb}{0.5,0.5,0.5}
\definecolor{codepurple}{rgb}{0.58,0,0.82}
\definecolor{backcolour}{rgb}{0.95,0.95,0.92}

\lstdefinestyle{mystyle}{
    backgroundcolor=\color{backcolour},
    commentstyle=\color{codegreen},
    keywordstyle=\color{magenta},
    numberstyle=\tiny\color{codegray},
    stringstyle=\color{codepurple},
    basicstyle=\ttfamily\footnotesize,
    breakatwhitespace=false,
    breaklines=true,
    captionpos=b,
    keepspaces=true,
    numbers=left,
    numbersep=5pt,
    showspaces=false,
    showstringspaces=false,
    showtabs=false,
    tabsize=2
}

\lstset{style=mystyle}

%==============================================================================
% Theorems and Proofs
%==============================================================================

\usepackage{amsthm}

\newtheorem{theorem}{Theorem}
\newtheorem{lemma}[theorem]{Lemma}
\newtheorem{proposition}[theorem]{Proposition}
\newtheorem{corollary}[theorem]{Corollary}

\theoremstyle{definition}
\newtheorem{definition}{Definition}
\newtheorem{example}{Example}

\theoremstyle{remark}
\newtheorem{remark}{Remark}

%==============================================================================
% TikZ Settings (for diagrams)
%==============================================================================

\usepackage{tikz}
\usetikzlibrary{arrows, positioning, shapes, calc}

% Common TikZ styles
\tikzset{
    block/.style = {
        draw,
        rectangle,
        minimum height=3em,
        minimum width=4em,
        align=center,
        fill=blue!10
    },
    sum/.style = {
        draw,
        circle,
        minimum size=6mm,
        fill=yellow!10
    },
    input/.style = {
        coordinate
    },
    output/.style = {
        coordinate
    },
    pinstyle/.style = {
        pin edge={to-,thin,black}
    }
}

%==============================================================================
% Tables
%==============================================================================

\usepackage{booktabs}  % For professional tables

% Custom table commands
\newcommand{\topline}{\hline\hline}
\newcommand{\midline}{\hline}
\newcommand{\bottomline}{\hline\hline}

%==============================================================================
% Bibliography
%==============================================================================

\newcommand{\citep}[1]{(\cite{#1})}
\newcommand{\citet}[1]{\citeauthor{#1} (\citeyear{#1})}

%==============================================================================
% Abbreviations for This Paper
%==============================================================================

% Model names
\newcommand{\ModelName}{TrajectoryErrorTransformer}
\newcommand{\PINN}{Physics-Informed Neural Network}
\newcommand{\FDM}{Fused Deposition Modeling}

% Dataset names
\newcommand{\DatasetTrain}{\texttt{train}}
\newcommand{\DatasetVal}{\texttt{val}}
\newcommand{\DatasetTest}{\texttt{test}}

%==============================================================================
% Page Layout (for IEEE format)
%==============================================================================

% Column width for IEEE format
\newcommand{\columnwidth}{3.5in}  % For single column figures
\newcommand{\doublecolumnwidth}{7in}  % For double column figures

% Figure width helper
\newcommand{\onecol}[1]{\begin{column}{\columnwidth}#1\end{column}}
\newcommand{\twocol}[1]{\begin{column}{\doublecolumnwidth}#1\end{column}}

%==============================================================================
% Misc Helpers
%==============================================================================

% Circled numbers
\newcommand{\circled}[1]{\tikz[baseline=(char.base)]{
    \node[shape=circle,draw,inner sep=1pt] (char) {\small #1};}}

% Checkmark
\newcommand{\checkmark}{\tikz\fill[scale=0.4, color=green!60!black](0,.35) -- (.25,0) -- (1,.7) -- (.25,.15) -- cycle;}

% X mark
\newcommand{\xmark}{\tikz\fill[scale=0.4, color=red] (0,0) -- (1,1) -- (0,1) -- (1,0) -- cycle;}

% Status indicators
\newcommand{\status}[2]{%
    \ifstrequal{#1}{pass}{\textcolor{green}{\checkmark}}%
    {\ifstrequal{#1}{fail}{\textcolor{red}{\xmark}}%
    {\ifstrequal{#1}{warn}{\textcolor{orange}{\textbf{!}}}%
    {#1}}}%
}

%==============================================================================
% Draft Watermark
%==============================================================================

\newcommand{\draftwatermark}{
    \begin{tikzpicture}[remember picture, overlay]
        \node[rotate=30, scale=10, text opacity=0.05] at (current page.center) {
            \textbf{DRAFT}
        };
    \end{tikzpicture}
}

% Uncomment to add watermark:
% \draftwatermark

%==============================================================================
% Print/Online Differences
%==============================================================================

% Use this to show different content for print vs online
\newcommand{\onlineonly}[1]{\ifmode{online}{#1}{}}
\newcommand{\printonly}[1]{\ifmode{print}{#1}{}}

%==============================================================================
% Common Math Shortcuts
%==============================================================================

% Derivatives
\newcommand{\ddx}[1]{\frac{d#1}{dx}}
\newcommand{\pdx}[1]{\frac{\partial #1}{\partial x}}
\newcommand{\ddt}[1]{\frac{d#1}{dt}}
\newcommand{\pdt}[1]{\frac{\partial #1}{\partial t}}

% Integrals
\newcommand{\intf}{\int_{-\infty}^{\infty}}
\newcommand{\intzero}{\int_{0}^{\infty}}

% Limits
\newcommand{\limn}{\lim_{n \to \infty}}
\newcommand{\limx}{\lim_{x \to \infty}}
\newcommand{\limzero}{\lim_{x \to 0}}

% Big parentheses
\newcommand{\bigp}[1]{\bigl(#1\bigr)}
\newcommand{\Bigp}[1]{\Bigl(#1\Bigr)}
\newcommand{\biggp}[1]{\biggl(#1\biggr)}
\newcommand{\Biggp}[1]{\Biggl(#1\Biggr)}

% Big brackets
\newcommand{\bigb}[1]{\bigl[#1\bigr]}
\newcommand{\Bigb}[1]{\Bigl[#1\Bigr]}
\newcommand{\biggb}[1]{\biggl[#1\biggr]}
\newcommand{\Biggb}[1]{\Biggl[#1\Biggr]}

%==============================================================================
% End of Custom Commands
%==============================================================================
